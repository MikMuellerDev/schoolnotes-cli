\section*{Was ist Linux}
\subsection*{Subsection}
\subsubsection*{Hallo}
\subsection*{NEU}

Als Linux oder GNU/Linux (siehe GNU/Linux-Namensstreit) bezeichnet man in der Regel freie, unixähnliche Mehrbenutzer-Betriebssysteme, die auf dem Linux-Kernel und wesentlich auf GNU-Software basieren. Die weite, auch kommerzielle Verbreitung wurde ab 1992 durch die Lizenzierung des Linux-Kernels unter der freien Lizenz GPL ermöglicht. Einer der Initiatoren von Linux war der finnische Programmierer Linus Torvalds. Er nimmt bis heute eine koordinierende Rolle bei der Weiterentwicklung des Linux-Kernels ein und wird auch als Benevolent Dictator for Life (deutsch wohlwollender Diktator auf Lebenszeit) bezeichnet.

Das modular aufgebaute Betriebssystem wird von Softwareentwicklern auf der ganzen Welt weiterentwickelt, die an den verschiedenen Projekten mitarbeiten. An der Entwicklung sind Unternehmen, Non-Profit-Organisationen und viele Freiwillige beteiligt. Beim Gebrauch auf Computern kommen meist sogenannte Linux-Distributionen zum Einsatz. Eine Distribution fasst den Linux-Kernel mit verschiedener Software zu einem Betriebssystem zusammen, das für die Endnutzung geeignet ist. Dabei passen viele Distributoren und versierte Benutzer den Kernel an ihre eigenen Zwecke an.

Linux wird vielfältig und umfassend eingesetzt, beispielsweise auf Arbeitsplatzrechnern, Servern, Mobiltelefonen, Routern, Notebooks, Embedded Systems, Multimedia-Endgeräten und Supercomputern. Dabei wird Linux unterschiedlich häufig genutzt: So ist Linux im Server-Markt wie auch im mobilen Bereich eine feste Größe, während es auf dem Desktop und Laptops eine noch geringe, aber wachsende Rolle spielt. Im März 2021 war es in Deutschland auf 2,19 \% der Systeme installiert.

Linux wird von zahlreichen Nutzern verwendet, darunter private Nutzer, Regierungen, Organisationen und Unternehmen.
