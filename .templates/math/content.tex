\begin{flushright}
    \textbf{siehe Buch S. XX}
\end{flushright}
\vspace{-1cm}
\section*{Titel}

% PLOT DEFINED BY COORDINATES
% \begin{tikzpicture}
%     \begin{axis}[
%             width=15cm,
%             height=10cm,
%             title={Aufzugsgraph},
%             xlabel={Zeit in $s$},
%             ylabel={Geschwindigkeit in $\frac{m}{s}$},
%             xmin=0, xmax=10,
%             ymin=-3, ymax=3,
%             % xtick={0,2,4,6,8,10,12,14},
%             % ytick={0,1,2,-1,-2},
%             xtick distance=1,
%             ytick distance=1,
%             legend pos=north west,
%             xmajorgrids=true,
%             ymajorgrids=true,
%             grid style=dashed,
%         ]

%         \addplot[
%             color=red,
%             % mark=square,
%         ]
%         coordinates {
%                 (0,0)
%                 (1,2)
%                 (4,2)
%                 (5,0)
%                 (7,0)
%                 (8,-2)
%                 (9,-2)
%                 (10,0)
%             };
%         \legend{$v_{Aufzug}$}

%     \end{axis}
% \end{tikzpicture}


\begin{tikzpicture}
    \begin{axis}[
            axis lines = left,
            width=10cm,
            % height=10cm,
            title={Titel},
            xlabel={$x$},
            ylabel={$f(x)$},
            % xmajorgrids=true,
            % ymajorgrids=true,
            % grid style=dashed,
        ]
        %Below the red parabola is defined
        \addplot [
            domain=-10:10,
            samples=100,
            color=red,
        ]
        {x^2};
        \addlegendentry{$x^2$}
        %Here the blue parabola is defined
        \addplot [
            domain=-10:10,
            samples=100,
            color=blue,
            colormap/cool,
        ]
        {0.1*x^3};
        \addlegendentry{$x^3$}


        \addpoint{black}{0,0}{95}
        \addlegendentry{Schnittpunkt}
    \end{axis}
\end{tikzpicture}

% Begin Equasion
\subsection*{Lösung}
\begin{align*}
     &                 & f(x) & = 0                 &  & |\,\text{Ansatz} \\
     & \Leftrightarrow & 0    & = 4x^3+8x^3+16x^2+8 &  & |\,\text{:4}     \\
     & \Leftrightarrow & 0    & = x^3+2x^3+4x^2+2   &  &                  \\
\end{align*}
% End Equasion

% Begin Explaination
\subsection*{Erklärung}

